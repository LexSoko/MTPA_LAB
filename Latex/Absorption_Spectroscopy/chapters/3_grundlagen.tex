%********VORAUSSETZUNGEN & GRUNDLAGEN*********
\section{Prerequisite \& Basics}
\label{sec:voraussetzungen-grundlagen}

\subsection{Laser Principles}
    % Explain the basic principles of lasers, including stimulated emission, population inversion, and coherence.
    % Discuss their importance in scientific and technological applications.

\subsection{Pulsed Lasers and Time-Bandwidth Limit}
    % Describe pulsed lasers, their advantages, and limitations.
    % Discuss the concept of time-bandwidth limit and its relevance.

\subsection{Dielectric Mirrors}
    % Explain what dielectric mirrors are and how they enhance laser performance.
    % Discuss their role in reflecting specific wavelengths.

\subsection{Second Harmonic Generation (SHG)}
    % Introduce SHG as a nonlinear optical process.
    % Explain how crystals generate light at twice the input frequency.
    % Discuss the importance of phase matching.

\subsection{Iodine Absorption}
    % Cover the absorption properties of iodine, including rotation, vibration, and temperature dependence.
    % Explain natural linewidth and temperature broadening effects.

\subsection{Beer-Lambert Law}
    % Introduce the Beer-Lambert Law and its relevance in spectroscopy.
    % Explain how it relates absorbance to concentration and path length.

\subsection{Grating Spectrometer}
    % Briefly mention grating spectrometers and their role in analyzing laser spectra.
    % Explain how diffraction gratings disperse light for spectral measurements.





\subsection{Uncertainty analysis}
\label{sec:unsichi}

In the entire protocol, unless otherwise specified, the maximum uncertainty method according to Eq.\ref{equ:groestunsicherheit} \cite{MMETH} is used for all error calculations.
For this purpose, the total differential of the outgoing equation is formed and the absolute values of the summands are multiplied by the uncertainty determined.
All statistical evaluations are subject to a statistical uncertainty according to the Student's t-distribution.
\begin{equation}
    \varDelta f = \biggl| \frac{\partial f}{\partial x_{1}} \cdot \varDelta x_{1} \biggl| + \biggl| \frac{\partial f}{\partial x_{2}} \cdot \varDelta x_{2} \biggl| + .... + \biggl| \frac{\partial f}{\partial x_{n}} \cdot \varDelta x_{n} \biggl|
    \label{equ:groestunsicherheit}
\end{equation}
\newpage