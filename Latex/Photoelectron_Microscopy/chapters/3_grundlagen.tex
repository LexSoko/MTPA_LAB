%********VORAUSSETZUNGEN & GRUNDLAGEN*********
\section{Voraussetzung \& Grundlagen}
\label{sec:voraussetzungen-grundlagen}

\subsection{Unsicherheitsanalyse}
\label{sec:unsichi}
Im gesammten Protokoll, sofern nicht anders angegeben, wird die Größtunsicherheitsmethode laut Gl.\ref{equ:groestunsicherheit} \cite{MMETH} für alle Fehlerrechnungen benutzt.
Hierzu wird das totale Differential der ausgehenden Gleichung gebildet und die Absolutbeträge der Summanden mit der ermittelten Unsicherheit multipliziert.
Alle statistischen Auswertungen werden mit einer statistischen Unsicherheit laut der Studentschen t-Verteilung behaftet.
\begin{equation}
    \varDelta f = \biggl| \frac{\partial f}{\partial x_{1}} \cdot \varDelta x_{1} \biggl| + \biggl| \frac{\partial f}{\partial x_{2}} \cdot \varDelta x_{2} \biggl| + .... + \biggl| \frac{\partial f}{\partial x_{n}} \cdot \varDelta x_{n} \biggl|
    \label{equ:groestunsicherheit}
\end{equation}
\newpage